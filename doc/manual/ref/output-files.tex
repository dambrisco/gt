\subsection{Outputted Files}
Gt outputs many files based on the grammar defined by the input file. Below 
we will explain the structure of the gt generated files. We will mostly show the
the interfaces of the files and explain the major functions of each.
%
%
\subsubsection{Util File: \textit{grammar\_name}\_util.mli}
The utility file contains functions that help gt keep track of the current
line number and filename. There are also helper functions that allows
gt to get the current position data and to print the line number and filename 
more easily.\\
\lstinputlisting[language=ML]{./ref/generated-funs/util.mli}
%
%
%
\subsubsection{Syntax File: \textit{grammar\_name}\_syntax.ml}
The generated syntax file defines types that are based on the given grammar file.
These types use the OCaml optional type when the corresponding grammar file uses
the gt's optional feature and OCaml list type when the grammar definition
uses the repetition feature of gt. Also, the The terminals that are not in the abstract syntax
tree only contain the position data information where as the terminals in the abstract syntax
tree contain the position data and string refering the terminal.\\
\lstinputlisting[language=ML]{./ref/generated-funs/syntax.mli}
%
%
\subsubsection{Parse File: \textit{grammar\_name}\_parse.ml}
The top of the parser file sets the lexical classes to their corresponding type, whether 
it be \textit{\_\_term\_not\_in\_ast\_\_} or \textit{\_\_terminal\_\_}. Then the productions
are set to their corresponding OCaml types. The parse file that is generated by gt always starts 
with the \textit{main} production which is of the OCaml type option. It will always be defined as being 
the start symbol of the grammar or end of file (eof). End of file is used in case the file inputed to 
the generated parser is empty. Each parser file also has a \textit{cur\_position} which 
returns the current position data at any given point when parsing. Gt transforms the gt grammar 
from EBNF to BNF to easily define the given grammar in OCamllex and OCamlyacc.\\
\lstinputlisting[language=ML]{./ref/generated-funs/parse.mli}
%
%
%
\subsubsection{Lexical Classes File: \textit{grammar\_name}\_lex.ml}
The generated lexical file contains the OCamllex definitions for the gt defined lexical classes --
the terminals of the gt defined context-free grammar.
\lstinputlisting[language=ML]{./ref/generated-funs/lex.mll}
%
%
%
\subsubsection{Graph-Viz File: \textit{grammar\_name}\_gviz.ml}
This function allows the user to generate a visual representation of the syntax tree from a parsable program for the
gt generated parser. The generated visual syntax tree can be viewed using Graph-Viz which is available for free on 
their website (http://www.graphviz.org/Download.php). 
\lstinputlisting[language=ML]{./ref/generated-funs/gviz.mli}
%
%
%
\subsubsection{Pretty Printer AST File: \textit{grammar\_name}\_ppast.ml}
Pretty printer AST allows gt to print a parsed input file's syntax trees. These functions allow the user to
print a textual representation of the syntax tree for the parsed file. \\
\lstinputlisting[language=ML]{./ref/generated-funs/ppast.mli}
%
%
%
\subsubsection{Equality File: \textit{grammar\_name}\_eq.ml}
The equality function takes in a pair and returns true if the two elements in the pair are equal, otherwise it returns false.\\
\lstinputlisting[language=ML]{./ref/generated-funs/eq.mli}
%
%
%
\subsubsection{Pretty Printer File: \textit{grammar\_name}\_pp.ml}
Pretty printer allows gt to print a parsed input file in a readable format. This function prints a newline
wherever there was one in the original file, but does not print out the comments of the input file. \\
\lstinputlisting[language=ML]{./ref/generated-funs/pp.mli}



